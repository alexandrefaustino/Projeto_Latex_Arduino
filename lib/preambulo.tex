%importação de pacotes

\usepackage[utf8]{inputenc}     % Acentuação direta
\usepackage{geometry}           %configurações de página
\geometry{a4paper,left=2.5cm, top=3.5cm, right=2.5cm, bottom=2cm} %margens
\usepackage[nodisplayskipstretch]{setspace}     % trabalhar com espaçamento: simples, 1.5, ou duplo
\usepackage[T1]{fontenc}         % Codificação da fonte em 8 bits
\usepackage[brazil]{babel}       % Pacote de idioma
\usepackage{mathptmx}            % Usa a fonte Times New Roman
\usepackage{graphicx}            % Inserir figuras
\usepackage{multirow, multicol, array}     % Múltiplas linhas e colunas em tabelas
\usepackage{xcolor, comment, enumerate, indentfirst} % trabalhar com gráficos

% Pacotes para matemática
\usepackage{amsmath, amsthm, amsfonts, amssymb, dsfont, mathtools}

%Pacotes de Estilos do documento
\usepackage{1-pre-textuais/estilosPreTextuais}
