%importação de pacotes

\usepackage[utf8]{inputenc}     % Acentuação direta
\usepackage[T1]{fontenc}         % Codificação da fonte em 8 bits
\usepackage[brazil]{babel}       % Pacote de idioma
\usepackage{mathptmx}            % Usa a fonte Times New Roman
\usepackage{graphicx}            % Inserir figuras
\usepackage{multirow, multicol, array}     % Múltiplas linhas e colunas em tabelas
\usepackage{xcolor, comment, enumerate, indentfirst} % trabalhar com gráficos
\usepackage[fontsize=11pt]{scrextend}           %Tamanho da fonte (10,12,14)
\setlength{\parindent}{0pt}      % Zero de indentação ou seja, sem tabulação


\usepackage{geometry}           %configurações de página/margens
\geometry{a4paper,left=2.5cm, top=3.5cm, right=2.5cm, bottom=2cm} %margens

\usepackage[nodisplayskipstretch]{setspace}     % trabalhar com espaçamento: simples, 1.5, ou duplo
\singlespacing                   % Para um espaçamento simples
%\onehalfspacing                 %Para um espaçamento de 1,5
%\doublespacing                  %Para um espaçamento duplo

% Pacotes para matemática
\usepackage{amsmath, amsthm, amsfonts, amssymb, dsfont, mathtools}

%Pacotes de Estilos do documento
\usepackage{1-pre-textuais/estilosPreTextuais}
\usepackage{1-pre-textuais/estilosPreTextuais}
