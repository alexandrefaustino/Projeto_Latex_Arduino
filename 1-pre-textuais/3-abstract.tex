A proposal was developed from basic electricity and electronic classes, and in parallel to language classes in C programming, with the aim of reproducing the knowledge of handling electronic components satisfactorily, as well as the Arduino microcontroller itself. Students carry out a survey of technical information on lamps, air conditioners present in the institution's environments and on the water supply and storage system. Some software was developed along with the Arduino platform, capable of managing or controlling the brightness of the chosen environments, electronic activation of pumps, according to the water level and automatic shutdown of the air conditioning units by means of sensors. Based on the proposals developed, the team had the opportunity to participate in the 3rd SIMPIF (IFPB Research, Innovation and Graduate Symposium). The following proposals were presented: Smart glasses for the visually impaired with arduino; Residential control center with arduino; Autonomous irrigation system with arduino.\par
As the Research activities developed synchronously with the event were of great importance for continuous improvement in the development of software and surgeries of new ideas for automation and resolution of the main proposal that is this research, providing important data that can be easily implemented in the structure again The IFPB campus is under construction in the city of Santa Luzia PB.

%Escreva as keywords do abstract - Atenção! uso ponto para separar
\keywords{Arduino. Programming language “C”. Building automation. Microcontrollers.}