School indiscipline sometimes finds a vast field through digital social networks. This way, this article presents
part of the results of the research “The Relationship Between School Indiscipline and the Use of Social Networks
in the perception of students of higher secondary technical-professional education courses of the Federal
Institute of Paraíba - Campina Grande Campus, approved by the IFPB INTERCONECTA Program, which is an,
exploratory, qualitative and quantitative cross - sectional study that aims to investigate students’ perceptions
of the first and second years of the IFPB - Campina Grande Campus high school technical courses about what
are the most used social networks by these participants and how they define appropriate and inappropriate
behavior in these networks. The results showed that whatsapp, instagram and facebook are the social networks
most used by the participants. They also showed that the ease of communication and interaction with other
users as the main appropriate use and virtual crimes as the most cited inappropriate action in Social Networks.
The results of this research suggest the possibility of preventive training for young people and their families
about the use of social networks and their consequences


\vspace{0.5cm}% espaçamento entre  o abstract e as palavras-chaves

%Escreva as keywords do abstract - Atenção! uso ponto para separar
\keywords{Indiscipline. Management. Social networks. Suitable use. Inappropriate use}