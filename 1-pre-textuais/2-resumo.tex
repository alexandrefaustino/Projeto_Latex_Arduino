% Escreva seu resumo e a as palavras-chave
% Atenção! cuidado para não apagar o comando \palavrasChave ao final do documento

A indisciplina escolar, por vezes, encontra um campo vasto nas redes sociais digitais. Desse modo, este artigo
apresenta parte dos resultados da pesquisa “A relação entre a indisciplina escolar e o uso das redes sociais
na percepção de estudantes de cursos técnicos integrados do Instituto Federal da Paraíba – Campus Campina
Grande”, aprovada pelo Programa Interconecta do IFPB, caracterizada como uma pesquisa de campo,
exploratória, qualitativa e quantitativa e de corte transversal, que tem como objetivo investigar a percepção
dos estudantes dos primeiros e segundos anos dos cursos técnicos integrados ao ensino médio do IFPB –
Campus Campina Grande sobre quais são as redes sociais mais utilizadas por estes participantes e como eles
definem comportamento adequado e inadequado nestas redes. Os resultados mostraram que o whatsapp, o
instagram e o facebook são as redes sociais mais utilizadas pelos participantes; apontaram, ainda, a facilidade
de comunicação e interação com outros usuários como o principal uso adequado e os crimes virtuais como
a ação inadequada nas redes sociais mais citadas. Os resultados desta pesquisa sugerem a possibilidade de
formação preventiva para os jovens e seus familiares sobre o uso das redes sociais e suas consequências.

\vspace{0.5cm} %espaçamento entre autores e-mail % espaçamento entre o resumo e as palavras-chaves

%Escreva as palavras chaves - Atenção! uso ponto para separar
\palavrasChave{Indisciplina. Gestão. Redes sociais. Uso adequado. Uso inadequado.}
