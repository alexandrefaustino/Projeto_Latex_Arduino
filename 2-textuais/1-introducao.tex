%\use columnbreak, para quebrar o final da 1º coluna

O estabelecimento de regras de convivência e de normas que norteiem as práticas pedagógicas nem sempre são suficientes de evitar problemas no processo de ensino-aprendizagem. Assim, Vasconcellos (2009) ressalta que o tema disciplina escolar é constante na vida de estudantes, pais, professores, gestores educacionais e na mídia, assim como representa o tema mais solicitado para a capacitação e a formação continuada de professores. Além do mais, dados da Pesquisa Internacional sobre Ensino e Aprendizagem apontam que, no Brasil, os professores usam 20 do tempo das aulas para gerir o comportamento do ambiente. Já a média dos outros 33 países participantes da pesquisa para este mesmo fim é de 13 (OECD, 2014). Esse fato destaca que a gestão do comportamento é imprescindível como ferramenta auxiliar no processo
de ensino-aprendizagem. Entretanto, percebe-se que dificuldades de ordem disciplinar ocorrem não somente de modo presencial, mas também nos ambientes virtuais, como, por exemplo, as redes sociais. Estas, por intermédio do progresso tecnológico da telefonia móvel, que propiciou o desenvolvimento dos smartsphones, e da popularização da internet, com a gratuidade de acesso, transformaram-se em espaços de socialização de muitos usuários.As Redes Sociais Digitais se apresentam, portanto, como um espaço informal no qual também podem ocorrer agressões, ameaças e intimidações que, consequentemente, interferem nas relações ocorridas nos espaços escolares, seja com estudantes, funcionários ou educadores. Além do mais, nota-se que estes ambientes são mais atrativos que o ambiente escolar, competindo, então, com o tempo de estudo. Devido à indisciplina estar presente no cotidiano educacional e sabendo-se que o comportamento indisciplinado precisa ser gerido pela coletividade da instituição escolar e que se fazem necessárias ações que englobem a prevenção e a intervenção sobre o comportamento dos atores escolares, incluindose também como campo de atuação da psicologia escolar, cabe investigar a percepção dos estudantes dos primeiros e segundos anos dos cursos técnicos integrados ao ensino médio do Instituto Federal de Educação, Ciência e Tecnologia da Paraíba (IFPB) – Campus Campina Grande sobre qual são as redes sociais mais utilizadas por estes participantes e como eles definem comportamento adequado e inadequado nestas redes.
precisa ser gerido pela coletividade da instituição escolar e que se fazem necessárias ações que englobem a prevenção e a intervenção sobre o comportamento dos atores escolares, incluindo se também como campo de atuação da psicologia escolar, cabe investigar a percepção dos estudantes dos primeiros e segundos anos dos cursos técnicos integrados ao ensino médio do Instituto Federal de Educação, Ciência e Tecnologia da Paraíba (IFPB) – Campus Campina Grande  sobre qual são as redes sociais mais utilizadas por estes participantes e como  eles definem comportamento adequado e inadequado nestas redes.
precisa ser gerido pela coletividade da instituição escolar e que se fazem necessárias ações que englobem a prevenção e a intervenção sobre o comportamento dos atores escolares, incluindo se também como campo de atuação da psicologia escolar, cabe investigar a percepção dos estudantes dos primeiros e segundos anos dos cursos técnicos integrados ao ensino médio do Instituto Federal de Educação, Ciência e Tecnologia da Paraíba (IFPB) – Campus Campina Grande sobre qual são as redes sociais mais utilizadas por estes participantes e como eles definem comportamento adequado e inadequado nestas redes.
precisa ser gerido pela coletividade da instituição escolar e que se fazem necessárias ações que englobem a prevenção e a intervenção sobre 