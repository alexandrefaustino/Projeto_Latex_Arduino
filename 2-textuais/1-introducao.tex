\section{Introdução}
%\use o comando "\columnbreak", para quebrar o final da 1º coluna, caso necessite
%Digite a baixo a sua introdução
A tecnologia está cada dia mais presente no nosso dia a dia, atualmente dispomos de vários dispositivos capazes de realizar tarefas que antes demandavam tempo e esforço, hoje conseguimos nos comunicar com pessoas que estão a quilômetros de distância somente através de um simples celular, além de outras comodidades inerente ao conforto necessidades das pessoas. Temos aparelhos de ar- condicionado em residências, reservatórios de água na maioria das casas, estudos EAD por meio de computadores e uma infinidade de lâmpadas prontas para serem usadas conforme o horário, mas o quanto de eletricidade esses dispositivos consomem? Durante o verão o ar-condicionado é o grande consumidor de energia elétrica, sendo o principal influenciador no aumento da conta de luz. Contudo, é comum encontrarmos salas e departamentos com aparelhos ligados sem a presença de pessoas, por um longo intervalo de tempo, elevando ainda mais o consumo de energia elétrica. Outro fato rotineiro é o uso de lâmpadas de potências acima do ideal, ou mesmo ligadas onde não há fluxo contínuo de pessoas, gerando desperdício de energia e elevando a conta de luz. Muitas residências usam caixas de água suspensas como reserva de água para uso diário, porém em muitas regiões a pressão da água distribuída não é suficiente para abastecer a mesma, ou como no caso da cidade de Santa Luzia em que existe racionamento de água. É possível usar sensores acoplados a caixa d’água que podem facilmente enviar informações diretas e precisas aos microcontroladores Arduino sobre o nível da água. Com o algoritmo correto, o sistema controlará o acionamento da bomba para o reabastecimento sempre que necessário, consequentemente, também será capaz de acionar o desligamento após o nível ideal desejado, evitando desperdício com água e energia. Com tais medidas podemos comparar o consumo do prédio antes, e depois da instalação dos dispositivos e verificar a potencialidade econômica gerada pela da automação na instituição. A cidade de Santa Luzia apresenta um alto índice de insolação anual, tendo um grande potencial para geração de energia elétrica por meio de placas solares. Sendo possível aumentar potencial econômico gerado pela automação e instalando placas solares, algo que pretendemos analisar durante o projeto.