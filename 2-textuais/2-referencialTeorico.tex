\section{Referencial Teórico}
%\use o comando "\columnbreak", para quebrar o final da 1º coluna, caso necessite
%Digite a baixo o seu referencial teórico
Para se pesquisar sobre a indisciplina discente no
espaço escolar é necessário compreender os diversos
pontos que a permeiam.
Reis et al. (2013) retratam uma pesquisa realizada
em 2011 no município de Contagem, Minas Gerais,
com uma amostra de 678 alunos, com idades entre
13 e 15, com o objetivo de analisar as vulnerabilidades
à saúde na adolescência, associadas às condições
socioeconômicas, às redes sociais, às drogas
e à violência.
Como resultado, obteve-se um percentual elevado
de adolescentes (40,4) beneficiado pelo Programa
Bolsa Família; 14,6 trabalhavam em seu tempo livre;
57,1 já haviam experimentado bebida alcoólica e
23,6 tabaco. Identificou-se, também, 15 de relato
de agressão e 26,7 de bullying. A grande maioria,
64,5, informou nunca/raramente conversar com
os pais sobre as dificuldades cotidianas, e 22 das
adolescentes relataram insônia e/ou sentimento de
solidão (REIS, 2013).
Em outro exemplo, ao analisarem um caso
de expressão de ódio nas redes sociais, Amaral
e Coimbra (2015) identificaram a ação de haters,
que se configuram como usuários do universo das
redes sociais que promovem violência e ódio nestes
ambientes. Nesta investigação, as autoras observaram
que haters, através de perfis fakes, posicionaram-se
contra a postagem da jornalista Nana Queiroz, que
publicou uma foto de topless em frente ao Congresso
Nacional para se solidarizar com uma campanha contra
à prática do estupro. Esses haters utilizaram memes e
agressões com palavrões (puta e vadia, por exemplo),
pelo fato de não concordarem com a exposição
corporal da jornalista, avaliando essa exposição como promíscua (AMARAL; COIMBRA, 2015).
{Corona tem cura?}
Percebe-se que a vida de muitos jovens
está permeada por vários desafios, portanto o
comportamento indisciplinado pode ter como causa
vários fatores. Assim, Gotzens (2003) descreve
algumas sugestões que servem como guia para o