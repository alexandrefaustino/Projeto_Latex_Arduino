\section{Método da pesquisa}
%\use o comando "\columnbreak", para quebrar o final da 1º coluna, caso necessite
%Digite a baixo o método da pesquisa

A relação entre indisciplina e o uso das redes
sociais na percepção de estudantes dos primeiros
e segundos anos dos cursos técnicos integrados ao
ensino médio do IFPB – Campus Campina Grande é
um tema pouco explorado. Dessa forma, a presente
pesquisa pode ser definida como exploratória, já que
tem como objetivo analisar um tema ou problema de
pesquisa pouco estudado ou sobre o qual se tenham
muitas dúvidas, além de ser um estudo que pretende
pesquisar sobre áreas e temas em outras perspectivas
(SAMPIERI; COLLADO; LUCIO, 2013).
Sobre a abordagem do problema, esta pesquisa
é caracterizada como qualitativa e quantitativa.
De acordo com Richardson et al. (2012), o método
quantitativo é caracterizado pela coleta de informações
e o tratamento destas por intermédio de técnicas
estatísticas, objetivando garantir a precisão dos
resultados, evitar distorções na análise e interpretação
dos dados e possibilitar uma margem de segurança
sobre as deduções. Esse método contribui, além
disso, para se descobrir e classificar a relação entre
as variáveis, assim como investigar a relação de
causalidade entre os fenômenos. No método qualitativo
procura-se entender a natureza de um fenômeno
social. Ressalta-se que caráter qualitativo também
está presente nos estudos estritamente quantitativos,
mesmo quando as informações foram transformadas
em dados quantificáveis (RICHARDSON et al., 2012).