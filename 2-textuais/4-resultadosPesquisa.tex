\section{Resultados da pesquisa}
%\use o comando "\columnbreak", para quebrar o final da 1º coluna, caso necessite
%Digite a baixo o método da pesquisa
\subsection{Redes sociais utilizadas pelos
estudantes}

A análise da questão 1 identificou que o instagram
é a rede social mais utilizada pelos estudantes
pesquisados (29,30%). O instagram2 é uma rede
social em que o usuário pode postar fotos e vídeos de
longa e curta duração. Pode-se também interagir com
publicações de outras pessoas através de comentários
e curtidas. O usuário pode seguir o outro para que
possa ficar por dentro de tudo que ele posta.
Whatsapp3 foi a segunda rede social mais citada
pelos estudantes, com 27,90%. Trata-se de um
aplicativo muito utilizado para a troca de mensagens
instantâneas. Além de disponibilizar o envio e
recebimento de arquivos de mídia, oferece a opção
de fazer ligação em chamada de voz e vídeo.
A terceira mais citada foi o Facebook 4, com
27,10%. Esta rede social é muito popular, pela
facilidade de se conectar com diversas pessoas do
mundo. O Facebook é utilizado para compartilhamento
de informações como fotos, vídeos, mensagens,
entre outros.Já o Twitter 5 foi citado pelos estudantes com percentual de 4,80%. Este é uma rede social em que
o usuário envia e recebe mensagens com até 280
caracteres de outros seguidores, atualizadas em
tempo real.